\documentclass[11pt, a4paper]{article}

\usepackage{fontspec}
\usepackage{geometry}
\usepackage{fancyhdr}
\usepackage[hidelinks]{hyperref}
\usepackage[normalem]{ulem}
\usepackage{multicol}

\geometry{
  top=2cm,
  bottom=2cm,
  left=2cm,
  right=2cm,
  marginparsep=4pt,
  marginparwidth=1cm
}

\renewcommand{\headrulewidth}{0pt}
\pagestyle{fancyplain}
\fancyhf{}
\lfoot{}
\rfoot{}

\setlength{\parindent}{0pt}
\setlength{\parskip}{0pt}

\usepackage{xunicode}
\defaultfontfeatures{Mapping=tex-text}

\begin{document}

\begin{center}
\textbf{LING297: Pragmatics --- Fall 2022}
\end{center}

\vspace{0.5cm}

% \textbf{Website}: \texttt{https://statsmaths.github.io/ling297-f22}

\textbf{Instructor}: Taylor Arnold, \url{tarnold2@richmond.edu}

\bigskip


\textbf{Meeting Time and Location}: Jepson 101, Wednesdays 4:30pm-5:45pm

\bigskip

\textbf{Credits}: 0.5

\bigskip

\textbf{Prerequisities}: None.

\bigskip

\textbf{Description}: This course gives an introduction to pragmatics,
a subfield of linguistics that studies language use in context. Pragmatics
is closely related to several other disciplines, including sociolinguistics,
cultural studies, and the philosophy of language. We will focus on
exploring three core ares of research within pragmatics: reference (the connection
between words and real-world objects/ideas), implicature (what is suggested by
context but not literally expressed), and speech acts (the use of language to
perform larger social actions).

\bigskip

\textbf{Textbook:}
Betty J. Birner, \textit{Introduction to Pragmatics},
Wiley--Blackwell (2012). 9781405175838.

\bigskip

\textbf{Format:}
This course will be run as a seminar. Readings and exercises
drawn from the above textbook will be assigned each week. During class meetings,
we will discuss the reading and answers to the practice problems. Grades will
be assigned based on participation and a one-page reflection due on the last
day of class.

\bigskip

\textbf{Topics:}
Below is a rough outline of weekly topics, with relevant sections of the textbook.
The exact pace and focus may change based on the interests of the class.

\begin{itemize} \setlength\itemsep{0em}
\item \textbf{Week 01}: Introduction
\item \textbf{Week 02}: Defining Pragmatics (1.1--1.4)
\item \textbf{Week 03}: The Cooperative Principal (2.1--2.2)
\item \textbf{Week 04}: Gricean Implicature (2.3--2.5)
\item \textbf{Week 05}: $Q$-, $I$-, and $M$-implicature (3.1)
\item \textbf{Week 06}: Relevance Theory (3.2--3.4)
\item \textbf{Week 07}: Reference I (4.1--4.3)
\item \textbf{Week 08}: Reference II (4.4--4.6)
\item \textbf{Week 09}: Presuppostion I (5.1--5.3)
\item \textbf{Week 10}: Presuppostion II (5.4--5.7)
\item \textbf{Week 11}: Speech Acts: Performative Utterances (6.1--6.2)
\item \textbf{Week 12}: Speech Acts: Felicity Conditions (6.3)
\item \textbf{Week 13}: Speech Acts: Direct and Indirect Speech (6.4--6.5)
\item \textbf{Week 14}: Speech Acts: Joint Acts (6.6-6.7)
\end{itemize}

\end{document}
